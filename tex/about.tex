% Official design documentation for Emerald

\documentclass{article}
\usepackage{authblk}
\usepackage{alltt}

\title{Emerald Design Documentation}
\author[1]{Bryce Davis\thanks{me@bryceadavis.com}}
\author[1]{Nick Diguido\thanks{nick@diguido.com}}
\affil[1]{Datum Unlimited}

\begin{document}

    \maketitle
    \pagebreak

    \renewcommand{\abstractname}{Notes}
    \begin{abstract}
        Emerald is currently pre-alpha. Things \emph{will} change drastically.
        Be prepared for things to just stop working randomly.
    \end{abstract}
    \pagebreak

    \part{Language Design}
        \section{Paradigms}
            \subsection{Imperative}
                Emerald includes most of C's basic control structures:
                \begin{itemize}
                    \item if
                    \item if-else
                    \item while
                    \item do-while
                \end{itemize}
                For-loops are not included since they can be implemented using
                while-loops, compound Boolean statements, and accumulators.
                Emerald also includes some interesting control structures from
                Ruby, including:
                \begin{itemize}
                    \item unless
                    \item until
                \end{itemize}
                This are the opposites of if and while, and their use can be
                determined from English usage.
            \subsection{Object-Oriented}
                Emerald is unique in the idea that while data and functions are
                both considered first-class, they are separate. Data are handled
                using \texttt{types} while functions are their own first class
                objects. Higher-level data types are combinations of built-in
                and user-defined types in a class-like format, like so:
                \begin{alltt}
                    \textbf{type} {
                        int data
                        string name
                    } sometype
                \end{alltt}
            \subsection{Functional}

\end{document}
